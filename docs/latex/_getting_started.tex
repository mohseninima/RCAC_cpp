This is a guide on getting started with the \hyperlink{class_r_c_a_c}{R\+C\+AC} library.

\section*{How to install? }

First download the source code from the \href{https://github.com/mohseninima/RCAC_cpp}{\tt repository}. Only the $\ast$.cpp and $\ast$.hpp files are needed. Next, you will need a copy of the Eigen matrix library, located \href{http://eigen.tuxfamily.org/}{\tt here}. Extract the {\itshape Eigen} and {\itshape unsupported} folders and place them in the same folder as the \hyperlink{class_r_c_a_c}{R\+C\+AC} code. You should have a folder structure similar to the following.


\begin{DoxyCode}
<root>
    Eigen/
    unsupported/
    RCAC.cpp
    RCAC.hpp
    RCACCreator.hpp
    RCAC*.cpp
    RCAC*.hpp
    .
    .
    .
    main.cpp (your code)
\end{DoxyCode}


\section*{Using Recursive Least Squares (R\+LS) \hyperlink{class_r_c_a_c}{R\+C\+AC} in C++ }

Using R\+LS \hyperlink{class_r_c_a_c}{R\+C\+AC} in your own code requires 6 main steps


\begin{DoxyEnumerate}
\item Including \char`\"{}\+R\+C\+A\+C\+Creator.\+hpp\char`\"{} and Eigen
\item Defining and assigning the \hyperlink{structrcac_rls_flags}{rcac\+Rls\+Flags} struct
\item Defining and assigning the \hyperlink{structrcac_filt}{rcac\+Filt} struct
\item Initializing \hyperlink{class_r_c_a_c}{R\+C\+AC}
\item Using the \hyperlink{class_r_c_a_c_a956bb6a557f050d3808d5392fd3add20}{R\+C\+A\+C\+::one\+Step()} method to propagate \hyperlink{class_r_c_a_c}{R\+C\+AC} by one step
\item Using the \hyperlink{class_r_c_a_c_ad93e5753d1810d7c3b2f6fbf56857a51}{R\+C\+A\+C\+::get\+Control()} method to get the associated control input
\end{DoxyEnumerate}

\subsection*{Creating the \hyperlink{structrcac_rls_flags}{rcac\+Rls\+Flags} struct }

This struct defines the base parameters such as controller order and sizes of the inputs and outputs. The list of members is
\begin{DoxyItemize}
\item lu\+: Number of control inputs (Type\+: int)
\item ly\+: Number of sensor measurements (Type\+: int)
\item lz\+: Number of performance measurements (Type\+: int)
\item Nc\+: Controller order (Type\+: int)
\item filtorder\+: Order of the filter $G_f$ (Type\+: int)
\item k\+\_\+0\+: Starting timestep for the controller (Type\+: int) (Usually = Nc)
\item P0\+: Initial value of the covariance matrix for R\+LS (Type\+: Eigen\+::\+Matrix\+Xd) (ltheta by ltheta)
\item Ru\+: Weighting matrix on the control input (Type\+: Eigen\+::\+Matrix\+Xd) (lu by lu)
\item Rz\+: Weighting matrix on the retrospective performance (usually identity) (Type\+: Eigen\+::\+Matrix\+Xd) (lz by lz)
\item lambda\+: The forgetting factor for R\+LS (Type\+: double)
\item theta\+\_\+0\+: Initial value for the controller coefficients (Type\+: Eigen\+::\+Vector\+Xd) (ltheta by 1)
\end{DoxyItemize}

The parameter ltheta is not needed to be given to \hyperlink{class_r_c_a_c}{R\+C\+AC} directly but is defined as $ l_\theta = N_c l_u(l_u+l_y) $

An example struct for a 2 input 4 output plant with a 10th order controller and filtorder of 4 is given below 
\begin{DoxyCode}
\hyperlink{structrcac_rls_flags}{rcacRlsFlags} FLAGS;
FLAGS.lu = 2;
FLAGS.lz = 4;
FLAGS.ly = 4;
FLAGS.Nc = 10;
\textcolor{keywordtype}{int} ltheta = FLAGS.Nc*FLAGS.lu*(FLAGS.lu+FLAGS.ly); \textcolor{comment}{//Define ltheta to help create other variables}
FLAGS.filtorder = 4;
FLAGS.k\_0 = FLAGS.Nc;
FLAGS.P0 = 100000*MatrixXd::Identity(ltheta, ltheta); \textcolor{comment}{//Create a matrix with diag values of 100000}
FLAGS.Ru = MatrixXd::Zero(FLAGS.lu, FLAGS.lu); \textcolor{comment}{//Create a zero matrix (no weighting on control effort)}
FLAGS.Rz = MatrixXd::Identity(FLAGS.lz, FLAGS.lz); \textcolor{comment}{//Create the identity matrix}
FLAGS.lambda = 1;
FLAGS.theta\_0 = MatrixXd::Zero(ltheta, 1); \textcolor{comment}{//Initialize the RCAC coefficients to zero}
\end{DoxyCode}


\subsection*{Creating the \hyperlink{structrcac_filt}{rcac\+Filt} struct }

\subsection*{Putting it all together }


\begin{DoxyCode}
\textcolor{preprocessor}{#include "Eigen/Core"}
\textcolor{preprocessor}{#include "\hyperlink{_r_c_a_c_creator_8hpp}{RCACCreator.hpp}"}
\textcolor{preprocessor}{#include <iostream>}
\textcolor{preprocessor}{#include <string>}
\textcolor{keyword}{using namespace }\hyperlink{namespace_eigen}{Eigen};

\textcolor{keywordtype}{int} main()
\{
    \textcolor{comment}{//Initialize the plant}
    \textcolor{keywordtype}{int} lx = 2;
    \textcolor{keywordtype}{int} lu = 2;
    \textcolor{keywordtype}{int} ly = 2;
    \textcolor{keywordtype}{int} lz = ly;

    MatrixXd A(lx,lx);
    A << 0.25,1,
          0,0.25;

    MatrixXd B(lx,lu);
    B << 1,0,
         0,1;

    MatrixXd C(ly,lx);
    C << 1,0,
         0,3;

    MatrixXd x(lx,1);
    x << 0,
         0;

    \textcolor{comment}{//End time of the simulation}
    \textcolor{keywordtype}{int} kend = 30;

    \textcolor{comment}{//Set Flags}
    \hyperlink{structrcac_rls_flags}{rcacRlsFlags} FLAGS;
    FLAGS.lu = lu;
    FLAGS.lz = lz;
    FLAGS.ly = ly;
    FLAGS.Nc = 4;
    \textcolor{keywordtype}{int} ltheta = FLAGS.Nc*FLAGS.lu*(FLAGS.lu+FLAGS.ly);
    FLAGS.filtorder = 4;
    FLAGS.k\_0 = FLAGS.Nc;
    FLAGS.P0 = 100000*MatrixXd::Identity(ltheta, ltheta);
    FLAGS.Ru = MatrixXd::Zero(FLAGS.lu, FLAGS.lu);
    FLAGS.Rz = MatrixXd::Identity(FLAGS.lz, FLAGS.lz);
    FLAGS.lambda = 1;
    FLAGS.theta\_0 = MatrixXd::Zero(ltheta, 1);

    std::cout << \textcolor{stringliteral}{"Flags Set\(\backslash\)n"}; 

    \textcolor{comment}{//Set Filter}
    \hyperlink{structrcac_filt}{rcacFilt} FILT;
    \textcolor{keywordtype}{int} Hnum = FLAGS.filtorder-1;
    FILT.filtNu.resize(FLAGS.lz, FLAGS.lu*(Hnum+1));
    FILT.filtDu.resize(FLAGS.lz, FLAGS.lz*Hnum);
    FILT.filtNu << 0,0, 1,0, 0.25,1, 0.0625,0.5,
                   0,0, 0,3, 0,0.75, 0,0.1875; 
    FILT.filtDu << 0, 0, 0, 0, 0, 0,
                   0, 0, 0, 0, 0, 0;
    FILT.filtNz = MatrixXd::Identity(FLAGS.lz, FLAGS.lz);
    FILT.filtDz =  MatrixXd::Zero(FLAGS.lz, FLAGS.lz);

    std::cout << \textcolor{stringliteral}{"Filter Set\(\backslash\)n"}; 

    \textcolor{comment}{//Initialize RCAC}
    std::string rcacType = \textcolor{stringliteral}{"RLS"};
    \hyperlink{class_r_c_a_c}{RCAC} *myRCAC;
    \textcolor{comment}{//Call the factory method to initialize RCAC}
    myRCAC = myRCAC->\hyperlink{class_r_c_a_c_af7b7133b676886d5010be725291c1a1d}{init}<\hyperlink{structrcac_rls_flags}{rcacRlsFlags}>(FLAGS, FILT, rcacType);
    std::cout << \textcolor{stringliteral}{"RLS Init\(\backslash\)n"}; 

    \textcolor{comment}{//Run Simulation}
    VectorXd y(FLAGS.ly,1);
    VectorXd z(FLAGS.lz,1);
    VectorXd u(FLAGS.lu,1);
    u = MatrixXd::Zero(FLAGS.lu, 1);

    \textcolor{keywordflow}{for} (\textcolor{keywordtype}{int} k = 0; k < kend; k++)
    \{
        y = C*x;
        z = y - VectorXd::Ones(FLAGS.ly);

        x = A*x + B*u;        

        myRCAC->\hyperlink{class_r_c_a_c_a956bb6a557f050d3808d5392fd3add20}{oneStep}(u, z, z);
        u = myRCAC->\hyperlink{class_r_c_a_c_ad93e5753d1810d7c3b2f6fbf56857a51}{getControl}();

        \textcolor{comment}{//std::cout << "y: " << y << ", z: " << z << ", u: " << u << std::endl;}
    \}
    \textcolor{keywordflow}{return}(0);
\}
\end{DoxyCode}
 